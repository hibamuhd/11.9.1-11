\documentclass[12pt]{article}
\usepackage{amsmath}
\usepackage{booktabs}
\usepackage{listings}
\usepackage{graphicx}

\title{Discrete Assignment-11.9.1-11}
\author{Hiba Muhammed \\
        EE23BTECH11026}
\date{} % Remove the date since it's auto-generated

\begin{document}

\maketitle

\section*{Problem Statement}
Write the first five terms in the sequence:
\begin{align*}
a_{0}  &= 3 \\
a_{n}  &= 3a_{n-1} + 2 \quad \text{for } n > 0
\end{align*}

\section*{Solution}
\begin{table}[h]
  \centering
  \caption{Input Parameters: First Term and General Formula}
  \begin{tabular}{|c|c|}
    \hline
    \textbf{Term} & \textbf{Value} \\
    \hline
    \(a(0)\) & 3 \\
    \(a(n)\) & \(3a(n-1) + 2\) for \(n > 0\) \\
    \hline
  \end{tabular}
\end{table}

Let's find the first 5 terms of the sequence:
\begin{align}
a(1) &= 3a(0)  + 2 = 3 \times 3 + 2 = 11 \\
a(2) &= 3a(1) + 2 = 3 \times 11 + 2 = 35 \\
a(3) &= 3a(2) + 2 = 3 \times 35 + 2 = 107 \\
a(4) &= 3a(3) + 2 = 3 \times 107 + 2 = 323 \\
a(5) &= 3a(4) + 2 = 3 \times 323 + 2 = 971 
\end{align}

So, the first 5 terms of the sequence are \(3, 11, 35, 107, 323\).

\section*{Solution using Z Transform}
\begin{align*}
A(z) &= \frac{2}{(1-z^{-1})(1-3z^{-1})} \\
&= \frac{A_1}{1-z^{-1}} + \frac{A_2}{1-3z^{-1}}
\end{align*}

Now, find the values of \(A_1\) and \(A_2\). Multiply through by the common denominator:
\begin{equation}
1 = A_1(1-3z^{-1}) + A_2(1-z^{-1})
\end{equation}

Equating coefficients, you can solve for \(A_1\) and \(A_2\):
\begin{align}
A_1 &= -1 \\
A_2 &= 3
\end{align}

Now, substitute these back into the modified partial fraction decomposition:
\begin{align*}
A(z) &= -\frac{1}{1-z^{-1}} + \frac{3}{1-3z^{-1}} \\
&= -\frac{1}{1 - z^{-1}} + \frac{3}{1 - 3z^{-1}}
\end{align*}

Now, you can find the inverse \(Z\)-transform of each term using the property \(Z^{-1}\left[\frac{1}{1-cz^{-1}}\right] = c^n u_n\). The result should be:
\begin{equation}
a_n = -u_n + 3(3^n u_n)
\end{equation}

\begin{figure}[h]
    \centering
    \includegraphics[width=0.7\linewidth]{11.9.1-11.png}
    \caption{Sequence plot generated from the Python script.}
    \label{fig:sequence-plot}
\end{figure}

\begin{figure}[h]
  \centering
  \includegraphics[width=0.8\linewidth]{11.9.1-11x.png} % Replace with the path to your saved image file
  \caption{Plot of the sequence \(a_n = -u_n + 3(3^n u_n)\)}
  \label{fig:sequence_plot}
\end{figure}

\end{document}

